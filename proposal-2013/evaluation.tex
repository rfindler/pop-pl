\section{Experimental Validation}

Our plan to evaluate the design of \poppl{} is to develop a program
(which we call POP-GDM) for the care and treatment of gestational
diabetes mellitus (GDM), and to conduct a simulation of a clinical
trial at Northwestern Memorial Hospital.

\subsection{GDM Background}

Due to rising maternal obesity and lower diagnostic threshold, GDM cases
will double to 600,000 women over the next 3 years. A crisis looms, as
resources are inadequate to manage this projected patient load. GDM
raises risk of stillbirth, C-section, and
pre-eclampsia~\citep{Langer2006} and has long-term effects, including
raised risk of impaired glucose tolerance in adult
offspring~\cite{Metzger2007}. The Hyperglycemia and Adverse Pregnancy
Outcome (HAPO) study of 25,000 pregnant women showed that adverse
outcomes in mother, fetus, and neonate rise with maternal glycemia at
24--28 weeks~\cite{Metzger2008}. A Swedish study found 20\% of
pregnant women at risk for GDM and a surprisingly low rate of GDM
screening~\cite{Persson2009}. Given similar trends in the US, an
opportunity remains for improvement of pregnancy outcomes through
higher efficacy interventions.

GDM treatment lowers perinatal morbidity and improves maternal quality
of life~\cite{Crowther2005}. There is a 2- to 4-fold increase in
metabolic complications and macrosomia with untreated GDM; treatment
eliminates this gap~\cite{Langer2005}. One of us was a PI for a
multi-center randomized control trial in women with mild GDM in the
24th--31st week of gestation~\cite{LandonPeaceman2009}. In this study in
958 women with abnormal glucose tolerance but fasting glucose $<$ 95
mg/dL, treatment reduced birth weight (3302 vs. 3408 g), neonatal fat
mass (427 vs. 464 g), birth weight $>$ 4000 g (5.9\% vs. 14.3\%),
shoulder dystocia (1.5\% vs. 4.0\%), Cesarean section (26.9\%
vs. 33.8\%), and combined gestational hypertension + preeclampsia
(8.6\% vs. 13.6\%; P=0.01). Based on these encouraging results and the
significant residuum of hyperglycemia seen in the treated cohort in
these clinical trails, we expect our POP-GDM program, with better
performance, will further improve outcomes.

\subsection{Simulation Facility Background}

A particular strength of this application is access to the unique,
separately funded, Medical Simulation and Immersive Learning Center
at the downtown Chicago campus of
Northwestern University. It provides simulation environments including
physical plant, electronic devices, and computers of (1) a 5-bed ICU,
(2) an ambulatory clinic, (3) a work area for interacting by phone,
text-messaging or email with patients at home, and (4) equipment and
support for producing high quality video for use in mastery learning
modules we will develop for clinicians and patients. We will use this
facility to conduct simulation trials of POP-GDM in successive cycles
of refinement and debugging. 

\subsection{Simulation Trial for POP-GDM}

Our simulation trials for POP-GDM will not involve real patients and
thus will not have the possibility of harming people.  That lack
aside, our simulation trials will be as realistic as possible. 

In particular, we plan to collect data from real patients, including
their diet, glucose readings, and (if necessary) insulin intake. 
%
We expect this data to be available in significant quantities as we
expect there to be 300 women with GDM to be seen in the Maternal-Fetal
Medicine Clinic at Northwestern Prentice over the 2 year enrollment
period for this proposal. 
%
We plan to use this data to perform pure computer-based simulations
that will help us determine how our POP-GDM program will respond to
real-world data.

We also plan to perform simulations with real clinicians, actor
patients, and simulated inputs from glucometers and accelerometers
based on the data collected from the patients.
%
We will assess timeliness as the duration between reception by the
clinician of a message from a device, a patient, or another clinician
and complete transmission of a response by the clinician to the
patient.
%
We will assess appropriateness according to concordance
between the action initiated by the clinician and what is called for
by the consensus guideline established by our expert panel. 
%
We will assess reliability according to latency and consistency. 

The directory of the Northwestern Simulation and Immersive Learning
facility and will provide assistance and oversight for these
simulation trials.

\subsection{Evaluation}

We hope to validate that POP-GDM improves adherence to prescribed
diet and activity in women at high risk for GDM. Depending on the
strength of the therapeutic effect, we may also show that POP-GDM is
superior to usual care for improving maternal glycemia and for
reducing macrosomia, and for avoiding insulin. 
