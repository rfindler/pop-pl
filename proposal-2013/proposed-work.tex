\section{Proposed Work}

\addtocounter{subsection}{-1}
\subsection{Understanding Best Practices}\label{sec:best-practices}

To begin, we will conduct an exhaustive search of the medical
literature, interviews of domain experts, clinical practice
guidelines, and many examples of prescriptions/protocols/institutional
practice guidelines so as to identify very fine-grained descriptions
of the ``medical algorithms'' used to manage GDM.

\paragraph{Expected outcomes.}
From this comprehensive survey, we expect to gain a better
understanding of the variety of approaches that clinicians use to
manage GDM. This survey process was previously developed by us for the
purpose of developing patient-oriented prescriptions for other medical
problems.  We expect this survey to serve as a crucial source of ideas
and insights as to how one might implement a successful \poppl{}
program for managing GDM and also to better understand how \poppl{}
programs will work in general.

\subsection{Event-based Programming}\label{sec:new-start}\label{sec:event}

From past work on POPA, we expect that the large-grain structure of a
\poppl{} program will describe how to react to various kinds of
events, in a general sense.  For example, consider another fragment of
the POPA program in figure~\ref{fig:popa}:
\begin{center}
\fbox{\scriptsize\begin{tabular}{ll}
\textit{Breakthrough pain:} &
\parbox[t]{5in}{If pain score is 8 cm or greater, INCREASE on-demand dose
  by (10) \underline{\hbox to .3in{}} micrograms; \underline{reassess} in 1 hour. Repeat 3 times. If pain
  score is 8 cm or greater on 3 \underline{consecutive} assessments then notify
  physician. } \\
\textit{Minimal pain:} &
\parbox[t]{5in}{If the visual analogue pain score is 2 cm or less on any 3 consecutive assessments,
REDUCE on-demand dose by (10) \underline{\hbox to .3in{}} micrograms.}
\end{tabular}}
\end{center}

This fragment covers two responses to a patient's pain scores; the
first covers the situation where the patient has increased pain and
the second when the pain decreases. In each case, the instructions are
guarded by a condition that spans some unspecified time.

\poppl{} programs consist of a series of interconnected fragments
based on real-world events, e.g., the patient indicating pain
severity, or a nurse noticing that a patient is unconsciousness, or
that a pulse oximeter detects that a patient's capillary blood O$_2$
saturation has fallen to a low level.

We expect to build on ideas from function reactive programming
languages~\cite{frp} and guarded commands~\cite{guarded-commands}.
Adapting these techniques to our setting poses interesting challenges,
however. In particular, tasks can compete for resources including
clinician time with other tasks in other prescriptions, and may
require human judgment, decision-making, and validation. For example,
POPA (Figure 1) begins by \emph{discontinuing} previously prescribed
drugs with potential for dangerous interactions with fentanyl, the
opioid used in POPA. In our vision of how these ``prescription
collisions'' will be handled in \poppl{} programs, each prescription
will specify a process to follow so as to resolve such collisions
instead of having a simple priority based approach with automatic
resolution.

A prescription is more like a reactive system in the domain of
embedded systems for controlling cars or airplanes. When an airplane's
engines fail, the response includes output of software on the
airplane, software on other airplanes, software in the control tower,
and humans in each of these settings.  The interactions are complex
and the stakes are high. Car and airplane software domains currently
require programmers with high expertise who usually work at a low
level. Fortunately, the time scale for executing a prescription is
orders of magnitude higher, and the available computing resources are
orders of magnitude greater. We will be able to concentrate more on
designing a simple and expressive language and less about how to map
the programming model to constrained devices. 

\paragraph{Research questions.}
What is the language that \poppl{} programmers should use to describe
events and event–patterns? How can we, given a \poppl{} program,
determine if all of the possible events are covered and there are no
redundant patterns? Lack of coverage could mean that important
information is being dropped (e.g., a patient's requests are being
ignored) and redundancy could mean that conflicting responses could
come from a particular stimulus.

\paragraph{Expected outcomes.}
We expect to develop new languages for describing event streams and
patterns of event streams, influenced by the way medical practitioners
and patients and devices respond to each other. In addition, we expect
to build new analyses that determine if a set of event patterns cover
all possible sequences of events and if a set of event patterns are
redundant.

\subsection{Iterative Refinement \& Debugging}\label{sec:iterative}

Iterative refinement~\citep{boehm:spiral} is crucial for any program
at the complexity of patient-oriented prescriptions and \poppl{})
programs. For example, in the initial version of the POPA program,
there was no Ibuprofen option (only Ketorolac and Acetaminophen
options). After some testing, it was discovered that Ibuprofen was
also needed and this initiated a refinement to the program. The team
collaboratively revised POPA over several iterations, then tested the
new prescription in simulation over several more iterations. Then, a
pilot study in a limited number of patients was conducted, after which
the change was released and used with all patients. After deployment,
statistical sampling and analysis was used to assure that the improved
version of POPA did not inadvertently introduce new problems in
routine practice.

Of course, iterative refinement is a standard part of software
engineering lore. What is different for \poppl{} programs is the
nature of the iterations---some parts require people to simulate
them. We expect that computational power can be used to increase the
efficiency and productivity of the \poppl{} software developer and of
the clinicians who perform the prescription.  Once prescriptions are
in a machine-executable form and input events are collected into a
machine-processable form, then variations on programs can be run on
the same inputs to check that a new version works as expected. That
is, the developer of a prescription needs a regression test suite and
new test cases to check new versions of the prescription.

\paragraph{Feasibility} 
Physicians and other prescribers do not have formal training in
computer programming but do have extensive experience writing actual
prescriptions. This prescription-writing experience is effectively a
parallel discipline that uses many constructs common to computer
programming but without the benefit of the extensive conceptual
framework provided by computer science.  Thus, our task is not to
create this language de novo but instead to create an intuitive
language for these prescribers that will be formally defined and more
robust, reliable, and powerful than free text.  To improve the
feasibility of iterative refinement, we plan to build a replay
mechanism into \poppl{} so that the stream of events from one run of a
program (including all of the actions taken by the patients and the
care-givers) can be used to simulate how a new \poppl{} program
runs. In the previous paper versions of POPs, manual review of medical
records and direct observation of the clinical process by humans was
used to gather information for purposes of debugging. Much of this
data–gathering can be automated for e-prescriptions, as the POP-SE
environment will capture a rich data-stream of clinical events.

\paragraph{Research questions.} How can we determine if the replay of
a \poppl{} program has failed because there is a bug in the program or
because some intended change to the program has invalidated the trace?

\paragraph{Expected outcomes.} We expect to define a notion of
well-formed trace and a compatibility notion between traces and
programs. Specifically, a well-formed trace will be compatible with a
\poppl{} program if and only if an error that results from running the
trace is due to a bug in the program.

\subsection{Staged Programming}\label{sec:staged-programming}

One of the most exciting directions for integrating computing into
medicine is in moving off of the desktop. Any given \poppl{} program
executes on a variety of machines and contains instructions for both
machines and humans. Consider, for example, this fragment of a
\poppl{} program that deals with one part of the daily glucose
monitoring of a GDM patient:

\begin{center}
\fbox{\scriptsize\begin{tabular}{l@{~~$\bullet$~}l}
\textit{Initial diagnosis:} &
\parbox[t]{5in}{Advance practice nurse tells the patient to measure
  the blood glucose level every morning between 5 and 8am.} \\
\textit{Daily monitoring:} &
\parbox[t]{5in}{Alert patient if no glucose level value received by 8:30} \\
&
\parbox[t]{5in}{Alert nurse if no glucose level value received on
  three subsequent days.}\\
&
\parbox[t]{5in}{If glucose value between 120 \& 140, patient to adjust
dose}\\
&
\parbox[t]{5in}{If value higher, consult advance practice nurse}\\
\end{tabular}}
\end{center}

This code fragment has machines talking to machines (the glucose meter
communicating with the cell phone, pager, or telephone of the patient
as well as communicating with the hospital to relay the data values),
machines talking to people (e.g., alerting the patient when no value
has been received), and people talking to people (the nurse informing
the patient of those parts of the prescription dependent on the
patient). This fragment is only a small fragment of the program.

In the same way that a playwright works on a single play and
distributes the ``work'' of the play to different actors, the
developer of a prescription wants to develop a single program rather
than separate programs for all of the different actors in its
execution. Precedents for splitting a single program into multiple
roles exist in the computing domain, including the problem of
splitting a web-based application into server-side and client-side
parts, which we have investigated previously~\citep{mfgkf:asej2004}.

In general, the problem is difficult, but usually because the problem
is splitting a traditional linear program into a concurrent system.
We expect \poppl{} programs to use an event-driven style that is
inherently concurrent and that more easily translates into real
devices and human actors. The problem is not as easy as extracting the
subset of a play that contains a single actor's lines, because events
will trigger cooperative responses from many different actors, but the
problem is still far easier than extracting concurrency from a
traditional single-threaded computation.

Additionally, \poppl{} programs can provide functionality that would
be highly desirable to clinicians. For example, clinicians must
simultaneously care for large numbers of different
patients. Interruptions are frequent. Setting priorities is
challenging. Mapping the urgency and importance of each task tasks
onto a Cartesian grid would give us the ability to provide a more
informative visualization of the available tasks, and thus permit
clinicians and patients to make more informed decisions. We expect
that providing this functionality would improve the effectiveness of
clinicians and would also diminish the stress inherent to a demanding
clinical environment.

\paragraph{Research questions.} How can we write \poppl{} programs so
that the roles of the various players (machines and people) are
separable? Can we automatically generate internally consistent views
of a \poppl{} program that are specialized for particular people and
understandable by those people? Similarly, can we automatically
generate internally consistent programs that run on the myriad of
devices given just a single \poppl{} program?

\paragraph{Expected outcomes.} 
We expect to build on ideas of staged
programming~\citep{multistageprogramming,
  efficient-program-specialization, runtime-generation-c,
  tickc:toplas} and aspect-oriented programming~\citep{aop} to come up
with a notion of staging that captures these very different parts of
\poppl{} program.

\subsection{Virtual Machines and Medical Devices}\label{sec:new-end}\label{sec:vm}

Following up on the problem of splitting a \poppl{} program into 
different pieces for different devices, we will need a platform
for actually running the pieces on the different devices and having
the devices communicate with each other successfully. We will design a
relatively high-level virtual machine language that is easily
retargetable to multiple different low-level, programmable devices.

Intermediate languages like the JVM~\citep{jvm} and the CLR~\citep{clr}
already provide guidance and even tool support for running on many
different programs, but few medical devices currently support running
such general purpose programs. We plan to develop a simple, highly–intuitive,
restricted language that is better suited to environments where
information must be carefully protected and to implement this language on various medical devices.

Our approach includes four steps 1) A threat and risk assessment
documenting the system components, the stakeholders, the security
requirements, and the threats against those requirements, 2)
Operational security guidelines describing countermeasures, technical
or otherwise, that mitigate security threats, 3) A policy document
describing goals, responsibilities, guidelines, and procedures, and
finally 4) A security architecture that accurately reflects the
security priorities and priorities for the prototype framework.  This
architecture will act to place \poppl{} in a secure sandbox isolating
the language from interactions with hardware devices and EMRs.

\paragraph{Research questions.} How can we restrict the computational
power of programs in our intermediate language to ensure that
information only goes to trusted recipients? 

\paragraph{Expected outcomes.} A new low-level language for medical
devices that guarantees security properties.

\subsection{Workflow Management Systems}

Workflow management systems are computer systems that monitor and
manage tasks performed by humans (and computers). These systems accept
the specification of some set of tasks and their dependencies and
outcomes and then accept input from humans to ensure that the proper
set of tasks occur in the proper order.

Most of these systems are too brittle for our needs, as they are not
generally programming languages, but instead GUI-based applications
that only allow partial specification of the tasks and their
dependencies, as such specifications quickly become too complex for
that kind of setting.

One notable exception is iTasks~\citep{itasks}. It is a
domain-specific language embedded in the programming language Clean
for the specification of tasks and their interactions. iTasks allows
programmers to specify workflows in a high-level, declarative manner
whereby tasks are composed out of atomic tasks and combinators that
express sequential composition or parallel composition (for
example). It also has excellent support for automatically building
GUIs for data entry based on type specifications. We expect to build
on the ideas in iTasks and specialize them to the medical setting.

\paragraph{Research questions.} One challenge in the medical
setting is that a specific task can change over time. For example, a
naive understanding of the directive ``take medication every 6
hours'' would lead to an inflexible timer that expects a patient to
either check a box or not at the pre-determined times. However, many different tasks compete for completion and many events potentially alter the precise timing of prescribed actions; further,
there is a notion of time built into that specification; taking the drug
dose 20 minutes early or late is probably fine (depending on the
medication and dosage guidelines), but if two hours have elapsed
since the correct time, then waiting for the next interval is
certainly the wrong behavior. Instead, the proper course might be to adjust the future drug dose schedule to catch up. The details of what these dose and schedule adjustments ought to be are often quite clear to the prescriber or pharmacist but are almost impossible to express clearly within the constraints of existing CPOE systems. So we need to understand how to include a notion of time into POP-PL prescribing models that allow a task to gracefully change over time.

We expect the core model of a prescription to be event-driven and we expect to apply the considerable expressive power of Racket to build prescription language constructs that intuitively and robustly express how medical tasks can be accomplished within an event-driven model.

\paragraph{Expected outcomes.} A notion of interoperability for
combining iTasks-like specifications with event-driven programming,
and a way to introduce a notion of time into iTasks.
