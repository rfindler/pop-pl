\documentclass[11pt]{article}
\usepackage{fullpage}
\usepackage{times}
\usepackage{graphicx}
\usepackage[round,sort]{natbib}
\usepackage{wrapfig}
\usepackage{caption}
\usepackage{boxedminipage}
\usepackage{url}
\usepackage{pdfpages}
\usepackage[compact]{titlesec}
\begin{document}

\renewcommand{\topfraction}{0.85}
\renewcommand{\textfraction}{0.1}
\renewcommand{\floatpagefraction}{0.75}

\newcommand{\note}[1]{\marginpar{\small{}\it{}#1}}

\newcommand{\ourtitle}[1]{\centerline{\begin{tabular}{c}\begin{tabular}{c}\bf{}SHF:Medium:Collaborative Research:\\
  \bf{}Domain-Specific Language Design for Patient-Oriented Prescriptions{}#1\end{tabular}\end{tabular}}}


\newcommand{\poppl}{POP-PL}
\newcommand{\beginwfig}[1]{\begin{wrapfigure}{r}{#1}}
\def\endwfig{\end{wrapfigure}}

\newcommand{\hashLang}{{\tt \#lang}}
\newcommand{\racket}[1]{{\tt #1}}
\newcommand{\OnScreen}[1]{{\sf #1}}
\newcommand{\atexp}{$@$-expression}

\newcommand{\ResearchQuestions}[1]{{\bf Research questions:} #1}
\newcommand{\ExpectedOutcomes}[1]{{\bf Expected outcomes:} #1}

\pagestyle{empty}


\noindent
\begin{center}
\begin{tabular}{c}
\textbf{\large Domain-Specific Language Design for Patient-Oriented Prescriptions} \\
Steve Belknap, MD; Robby Findler, CS PhD; Matthew Flatt, CS PhD
\end{tabular}
\end{center}

~

Medical errors in hospitals cause 2 to 4 million patient injuries and
440,000 deaths in US hospitals annually; this amounts to one-sixth of
annual US deaths~\citep{Kohn1999,James2013}.
%
Large studies and systematic reviews have found that healthcare
information technology fails to reliably reduce patient injuries or
deaths \citep{Garg2005,Linder2007,Zhou2009,Romano2011}.
%
Despite extensive effort and the expenditure of billions of dollars,
computerization has failed to solve this problem~\citep{Landrigan2010}.
%
The PIs attribute this failure to a pervasive misunderstanding of the nature of
computation in healthcare.
%
While there has been a computer technology transfer in healthcare, we
await an intellectual transfer, in which software design,
maintenance, and debugging unlock the full potential of computer
science to improve healthcare.

Others have noticed this problem and have responded by trying to
introduce computers to the process of prescribing medicine to
patients, using systems collectively dubbed Computerized Prescriber
Order Entry (CPOE).
%
When carefully implemented, CPOE has been shown to reduce the rate of
medication prescription errors.
%
Clinically-relevant benefits, however, such
as improving clinical outcomes or preventing patient deaths or
injuries, have not yet been clearly demonstrated~\citep{vanRosse2009}.
%
Even in terms of error prevention, CPOE's introduction to the hospital
has been a mixed blessing.
%
Certain medical errors, such as those related to poor handwriting, are
reduced, but other familiar errors continue to occur, and some new types
of error are introduced by CPOE.
%
In one report, CPOE actually \emph{caused} 22 types of medication
error; problems included fragmented display of drugs, inventory
display mistaken for dose guidelines, duplicate and incompatible
orders, and inflexible formats~\citep{Koppel2005}.

We have a different approach, based on the claim that
\begin{center}
\textbf{a prescription is a program}
\end{center}
That is, a prescription requires ``a series of definite steps that
carry out a procedure'' or, in other words, an \emph{algorithm}.
%
The research areas of programming languages and software engineering
have amassed substantial knowledge about how to accurately implement
an algorithm as a program, and for dealing with the issues of clearly
expressing and maintaining those programs.
%
This knowledge is directly
relevant to healthcare prescriptions.
%
In short, computers are not the missing piece to fix the problems of
medicine delivery, the \emph{intellectual contribution of computer
  science itself} is the missing piece.

We have applied this insight to significantly reduce the risk of severe
injury from medical use of opiods among hospitalized adults at
St.\ Francis Medical Center in Peoria, IL, a 731-bed tertiary care
academic medical center~\citep{Belknap2008}. During a 1 year base-line
period, there were three to seven severe or fatal opioid-associated
adverse drug events (ADEs) each month. After widespread adoption of
POPA, his prescription, the rate of severe/fatal opioid ADEs dropped
to zero/month. 

To validate the success of \poppl{} program, we also
conducted a statistical process control clinical trial and nested
subcohort analysis in a population of 153,260 hospitalized adults. In
the orthopedics subcohort, POPA increased recording of pain scores
(94\% vs.\ 72\%, P $<$ 0.00001) and use of adjuvant analgesics (95\%
vs.\ 40\%, P $<$ 0.00001) and resulted in fewer opioid-associated
severe adverse drug events than routine patient-controlled analgesia
(PCA) (0\% vs.\ 2.7\%, number needed to treat (NNT) = 35, P $<$
0.015). Hospital-wide, POPA use increased to 62\% of opioid
prescriptions (diffusion half-life = 98 days), while opioid-associated
severe/fatal adverse drug events fell from an initial peak of seven
per month to zero per month during the final 6 months (P $<$ 0.0016)
of the study.

The success of the POPA was due in part to the rigor that algorithmic
thinking brought to the process of building the prescription and in
part to extensive debugging. Our belief is that the methods and
technology of software engineering and programming languages can
replicate this success on a wide-scale.

To adapt software engineering techniques to implement and maintain
highly reliable \emph{electronic} prescriptions, Belknap, Findler, and
Flatt have been collaborating on the designing a Patient-Oriented
Prescription Programming Language (\poppl{}). 

We think we can be successful by tackling the following projects, but
we believe our approach can help anywhere that medical error rears its
ugly head and we welcome advice and guidance.

\newcommand{\aim}[1]{\vspace{.1in}\noindent\textbf{Aim #1}\\}

\aim{1: Identify, Define, and Annotate Current Algorithms for
  Management of DM2 or Other Chronic Medical Conditions Relevant
  to the Medical Care of Active Duty Personnel and Military Retirees}
 Information sources will include medical standard procedure manuals
 and training materials, direct observation of clinical encounters,
 interviews of patients and prescribers, discussion with senior
 leaders in military medicine and at TriCare, and analysis of clinical
 data to precisely define the current methods for managing chronic
 illness and to identify strengths, flaws, and opportunities for
 automating and improving these methods. The chief work product of
 this aim will be a detailed, algorithmic description of how DoD
 patients with chronic illness are currently managed. This detailed
 description of the current process will then inform design of data
 structures and data flows for other POPs (Aim 2) and for building and
 evaluating POP-DM2 (Aim 3).

\aim{2: Design, Build, and Test Data Structures and Data Flows for POP-DM2 or other POPs}
 POP-DM2 will rely on a refactoring of the information flow and the
 control structure typical in clinical practice. One barrier to
 improving DM2 care is lack of integration of patient-generated data,
 including medication adherence, glucometer readings, diet, and
 physical activity. This orphan data is poorly integrated into the
 medical record due to data silos, incompatible data formats, data
 transfer restrictions, or inadequate data capture. We will design and
 build POP-DM2 data structures for this patient-generated data. We
 will also evaluate data flows between POP-DM2 and personal
 technology, including glucometers, activity monitors, and smartphone
 apps. 

\aim{3: Develop and Evaluate a Patient-oriented Prescription for
 Diabetes Mellitus 2 (POP-DM2)}
 We will design, build, and debug POP-DM2--an e-prescription that will
 effectively integrate physicians, other clinicians and information
 technology to manage a patient's drug therapy, diet, exercise, health
 behaviors, and other aspects of DM2. POP-DM2 will include components
 for efficiently capturing data from glucometers, accelerometers,
 web-based diet diaries, and smart phones so as to monitor
 (respectively) blood glucose, physical activity, diet, and adherence
 to medication. The POP control structure will provide consistent
 monitoring and relentless, gentle nudges to help the patient adhere
 with the agreed plan. We will first conduct simulation trials and
 then a statistical process control clinical trial in actual
 patients. Evaluation will include adherence to drug, exercise, and
 diet prescriptions, adequacy of blood sugar control, and adverse drug
 events. We hypothesize that POP-GDM will outperform usual care in
 timeliness, appropriateness, and reliability.

\newpage

\bibliographystyle{plainnat}
\bibliography{bib}

\end{document}
