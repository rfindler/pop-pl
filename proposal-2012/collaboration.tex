
\section{The Team}

This proposal pulls together expertise in a wide range of areas, from
programming languages in computer science to clinical pharmacology in
medicine, to statistics. Each team member has a well-defined role to
play that builds on their past experience and success.

\begin{itemize}
\item\textbf{Robby Findler}; Findler is a Ph.D. in computer science;
  his dissertation improved the state of the art of software
  specifications, specifically how to state and enforce behavioral
  software contracts in a higher-order setting. He is one of the
  four core members of the Racket development team and has helped to
  design and build many of the domain-specific languages discussed in
  section~\ref{sec:racket} of the proposal. Findler will be leading
  the development of the domain-specific language, focusing on the
  requirements gathering and language design issues.

  \item\textbf{Steven Belknap}; Belknap is an M.D. specializing in
    Internal Medicine and Clinical Pharmacology. Belknap has extensive
    experience providing medical care hospitalized adults.  He also
    has training and experience as a computer programmer. His
    long–standing research interest has been ``algorithmic medicine''.
    He is the first to have successfully applied software engineering
    and debugging methods to prescriptions for management of asthma,
    aminoglycoside antibiotic treatment of infections, anticoagulation
    therapy, and severe pain. Dr.\ Belknap was the principal
    investigator of a study involving more than 50,000 hospitalized
    adult patients treated with opioid analgesics as discussed in
    section~\ref{sec:pap}. Belknap introduced the main idea in this
    proposal, namely that ``A Prescription is a Program''. Belknap
    provides the overall vision and leads the clinical part of this
    project.

\item\textbf{Ian Brooks}; Brooks is a Ph.D. in biochemistry
  (computational biophysics). He has worked as a research scientist
  for the pharmaceutical manufacturer Smithkline Beecham, and as a
  software designer and group leader for Wolfram Research, the
  publisher of Mathematica. He is currently Director of Health
  Sciences at the National Center for Supercomputing Applications at
  the University of Illinois. Brooks and Belknap have been
  collaborating on Patient–oriented Prescriptions for more than 12
  years and Brooks co-authored the 2008 article describing
  POPA. Brooks will lead the efforts to build POP-SE by repurposing
  existing NCSA software technology.

\item\textbf{Matthew Flatt}; Flatt is a Ph.D. in computer science; his
  dissertation improved the state of the art in modularity constructs
  for programming languages, specifically he designed, build, and
  proved properties of a new technical for mutually referential
  module systems. He is also one of the four core members of the
  Racket development team and has helped to design an build many of
  the domain-specific languages discussed in section~\ref{sec:racket}
  of the proposal. Flatt will be leading the development of the
  runtime system to support \poppl{}, focusing on the back-end and
  infrastructure support at the Racket level.

\item\textbf{Dennis West}; West is a Professor of Dermatology and
  Pediatrics, a pharmacist, and Director of the Dermatopharmacology
  Program in the Department of Dermatology at Northwestern University,
  one of the largest such clinical research programs in the U.S. He is
  also Chair for Administrative Review with the Institutional Review
  Board for the Office for the Protection of Research Subjects at
  Northwestern University.  He served as Chair for the Dermatology
  Expert Panel at the United States Pharmacopeia from 2005--2010. West
  also leads RADAR (Research on Adverse Drug Events And Reporting), a
  proactive pharmacovigilance collaboration based at Northwestern, and
  has worked closely with Dr.\ Belknap for several years within RADAR
  and other pharmacovigilance projects. His role on this project will
  involve coordination with Walgreens and with the Pharmacy Department
  at Northwestern Memorial Hospital, and analysis and dissemination of
  results, with particular focus on the prescription design of the
  project.

\item\textbf{Boyd Metzger} Metzger is a physician specializing in
  Internal Medicine and Endocrinology. He has had a career-long
  interest in the perinatal and long-term consequences of GDM, the
  detection and diagnosis of GDM and its treatment. Metzger is a
  leading authority in the field of GDM. He led the seminal HAPO
  study~\citep{hapo}, which helped demonstrated the need for a lower
  threshold for intervention in GDM. He will provide domain expertise
  and advice in the area of GDM.

\item\textbf{Charlotte Niznik}; Niznik is an RN, Advance Practice
  Nurse and is the Maternal Fetal Medicine Specialist who coordinates
  the GDM program and the Maternal Obesity program at
  Northwestern. She has collaborated in preparation of this grant
  application and is a key source of insight and experience regarding
  the management of GDM.

\item\textbf{Alan Peaceman}; Peaceman is the Chief of the Division of
  Maternal/Fetal Medicine at Northwestern University Feinberg School
  of Medicine.  He currently serves as Principal Investigator for the
  National Institute of Child Health and Human Development (NICHD)
  Maternal-Fetal Medicine Units Network Grant. Peaceman will
  collaborate on creation of POP–GDM and will provide domain expertise
  and clinical advice regarding management of GDM.

\item\textbf{Bonnie Spring}; Spring is a Ph.D. clinical health
  psychologist. Her research aims to understand the biobehavioral
  mechanisms that maintain unhealthy behaviors and to develop and test
  interventions that promote healthy behavior change. She has many
  years of NIH- and VA-funded clinical trials experience intervening
  to promote healthy behavior changes in diet, physical activity, and
  smoking, singly or conjointly. Spring sees the proposed research as
  an important next step in understanding how to optimize multiple
  risk behavior change.

\item\textbf{John Vozenilek}; Vozenilek is an Emergency Medicine
  Physician and is the Director of Simulation Technology and Immersive
  Learning for the Feinberg School of Medicine. In this capacity, he
  provides central coordination and oversight for undergraduate,
  graduate, interdisciplinary, and continuing medical education
  programs. For this project, Vozenilek will provide advice and
  direction regarding the performance of the prescription simulation
  portions of this proposed work. He will also participate in data
  analysis and preparation of manuscripts.

\item\textbf{Susan Eller}; Eller is a nurse and is the Director of
  Interprofessional Education at Northwestern University Feinberg
  School of Medicine. She will provide oversight and assistance
  regarding the performance of the prescription simulations.

\item\textbf{Paul Yarnold} Yarnold is a statistician. He is a
  long-standing member of the RADAR Project and a long–standing
  collaborator of Dr.\ Belknap. He is also a co-author of the article
  describing POPA. Yarnold was a member of the Northwestern faculty
  for many years He recently became the President of Optimal Data
  Analysis, inc. He is the co-creator of the Hierarchical Optimal
  Discriminant Analysis Methods~\citep{Yarnold2004} that we will use
  for statistical analysis during this project. Yarnold will
  participate in the analysis of prescription performance data,
  epidemiology data, and will work closely with our programmer and
  with Dr.\ Belknap for purposes of data analysis. He will collaborate
  in the writing of journal articles and presentations of the results.
\end{itemize}

\section{Meetings}

We expect to have quarterly, day-long, in-person meetings to ensure
that all of the team members understand what each other is doing and
can offer feedback. We plan to schedule these meetings in a manner
similar to a small conference or workshop, where participating members
will each have a 30 minute slot to present what they have accomplished
and what their plans are for the next quarter. While these kinds of
structured meetings would certainly be overkill for a project that was
just within a single discipline, the wide range of areas represented
here means that there are significant cultural and linguistic gaps
that must be bridged in order to collaborate effectively. Thus, we
expect a significant aspect of these longer, routine presentations to
lead to questions and discussions that help to bridge these gaps and make our
collaboration effective.

In addition, we also plan to hold regular meetings scheduled not by
the calendar, but by the development progress. That is, after a fixed
number of updates to the software repository, we will hold a
(virtual) meeting to discuss the latest changes to the software. In
these meetings we will do code-walks and discussions of the latest
changes to the software. We expect these meetings to happen roughly
weekly and to last about 30 minutes each.

The key personnel on this proposal are mostly concentrated in the
Chicago area, with two exceptions: Brooks is in Urbana-Champaign, and
Flatt is in Salt Lake City. Brooks routinely makes visits to Chicago
and Findler and Flatt have been collaborating across this distance for
nearly a decade (their collaboration itself began a decade before
that). Nevertheless, it is important to have regular team-wide
meetings. Accordingly, we have budgeted money in the Utah and NCSA
portions of the proposal to cover travel to Chicago (these trips are
less expensive than one might think for Flatt, as he tends to stay
with Findler and thus only needs to cover airfare).

\section{Software Infrastructure}

The Racket development team is already scattered around the United
States and has been able to exploit the internet to effectively
facilitate that collaboration at the level of the software.  In
particular, we have an many active email conversations, we have our
own \texttt{git} server\footnote{\url{http://git.racket-lang.org}}, our own
bug repository software\footnote{\url{http://bugs.racket-lang.org}}, and our
own integration testing\footnote{\url{http://drdr.racket-lang.org}} that
rebuilds and re-runs the entire test suite on each change to the
software repository.

We expect to be able to exploit this already existing infrastructure when
building \poppl{}. The development of our other domain-specific
languages has benefited tremendously by this infrastructure. More
specifically, our \texttt{git} server will help us keep multiple,
simultaneous changes to the code in sync; the bug reporting mechanism
helps us keep the open issues organized as development goes on; and
the integration testing software will help ensure that we receive
timely notification when changes to one part of the system require
changes to another part.

