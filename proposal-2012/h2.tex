\subsection{Establish Whether or Not POP-GDM Improves Simulated  Clinical Decision-making.}\label{sec:popgdm}
We will write \& debug POP-GDM---a \poppl{} program that manages a
patient's drug therapy, diet, exercise, health behaviors, \& other
aspects of GDM. We will conduct simulation trials to compare POP-GDM
to usual care. We hypothesize that POP-GDM will outperform usual care
in timeliness, appropriateness, \& reliability of clinical decisions.

The objective of this aim is to do requisite design \& debugging of
POP-GDM in a simulation environment so as to achieve this goal of
safety and usability prior to deployment in a clinical setting. Our
working hypothesis is that POP-GDM will outperform usual care in
timeliness, appropriateness, \& reliability of clinical decisions. We
will test this hypothesis among experienced clinicians with clinical
scenarios in Northwestern's state-of-the-art clinical simulation
facility. As in our prior work applying software design principles \&
debugging methods to paper-based prescriptions, we will test \& refine
POP-GDM over several development cycles until
reaching a safe, debugged version. We will then compare the timeliness, appropriateness, and
reliability of clinical decisions made using POP-GDM with decisions
made with usual care. Upon completion of these initial development
cycles, we will have a version of POP-GDM suitable for testing in
patients. Additional simulation testing will be done in parallel with
testing in real patients during later cycles of development of
POP-GDM.

\paragraph{Justification \& Feasibility}
The physician investigators for this project have in aggregate more
than seventy years of clinical experience in academic medicine. We
have done extensive patient safety work in our own institution and
nationally. Also, our previous work developing a patient-oriented
prescription for analgesia was focused heavily on patient safety; we
will again emphasize safety in this project. Prior to initiation of
this project we expect to acquire an extensive library of pre-printed
order sheets, protocols, institutional policies, \& also to identify
relevant clinical practice guidelines for management of GDM. Combined
with our own clinical experience, these resources will be used to
generate the content of POP-GDM. A particular strength of this
application is access to the unique, separately funded, brand-new
Simulation and Immersive Learning Facility at Northwestern University
providing simulation environments including physical plant, electronic
devices, and computers of (1) a 5-bed ICU, (2) an ambulatory clinic,
and (3) a work area for interacting by phone, text-messaging or email
with patients at home. We will use this facility to conduct our
simulation trials of POP-GDM. 
Given previous similar studies, we are confident of a ready
pool of clinician–volunteers sufficient to conduct these simulations, even
allowing for ``Lasagna's law'' \citep{Wouden2007}.

\paragraph{Research Design}
The approach we will use to accomplish this aim is similar to the one
we previously used for developing \& testing patient-oriented
prescriptions~\citep{Belknap2008, GraumlichBelknap2000} \& that others
used to develop checklists for surgery~\citep{HaynesGawande2009} \&
prevention of catheter-associated bloodstream infection~\citep{Pronovost2006}.

\paragraph{Creation of POP-GDM}
We will use a six-member panel~\citep{Libby2010} of willing clinicians with
expertise in managing GDM. We will query for \& distribute relevant
materials suggested by the panel \& also provide to them our library
of pre-printed order sheets, protocols, and clinical guidelines. We
will then use a modified email-based Delphi method in successive
rounds to achieve consensus as to the clinical tasks \& activities
that will constitute POP-GDM. These will be described in plain English
\& in the pseudocode that clinicians naturally use to describe their
prescriptions. For example, a guideline on diagnosis of GDM coauthored
by one of us~\citep{Metzger2010} contains the remarkable, quite
algorithmic, pseudocode shown in figure~\ref{fig:gdm-alg}.

\beginwfig{5.0in}
\vspace*{-.3in}
\begin{boxedminipage}{5.0in}
\vspace*{-.2in}
\parbox{5.0in}{\small
\begin{tabbing}
\textbf{First prenatal visit:}\\
Mea\=sure FPG, A1C, or random plasma glucose on all or only high-risk women. \\
  \> If \= results indicate overt diabetes as per Table 1\\
  \> \> Treatment and follow-up as for preexisting diabetes\\
  \> If results not diagnostic of overt diabetes\\
  \> \> and \= fasting plasma glucose $\geq$ 5.1 mmol/l(92 mg/dl) but $<$ 7.0 mmol/l(126 mg/dl), \\
  \> \> \> diagnose as GDM.\\
  \> \> and fasting plasma glucose $<$ 5.1 mmol/l (92 mg/dl),\\
  \>  \> \> test for GDM from 24--28 weeks gestation with 75-g OGTT
\end{tabbing}}\vspace*{-.3in}
\caption{GDM Guideline~\cite{Metzger2010}, written as an algorithm}\label{fig:gdm-alg}
\end{boxedminipage}
\vspace*{-.2in}
\endwfig

We will translate into \poppl{} the expert panel's consensus
recommendations on GDM functionality \& debug this code in our
simulation laboratory. POP-GDM will include redundancies to handle
problems such as malfunctioning tech, lost messages, \& flawed or
delayed task execution. As part of this evaluation of POP-GDM, we will
evaluate the dietary assessment tools (smartphone-based and web-based)
that we will use to monitor diet, the accelerometers we will use to
measure physical activity, \& the glucometers we will use (when
indicated) to monitor patient glucose. We will evaluate suitability of
various glucometers \& accelerometers for this project and
exhaustively test, debug, \& optimize information transfer among
devices, patients, \& clinicians.

A growing body of research indicates that most patients have
difficulty adhering to dietary and physical activity
recommendations. Traditional paper methods for monitoring physical
activity \& diet are prone to retrospective bias, and overestimate
behavior adherence~\citep{Stone2002}. Paper records also fail to yield
timely feedback to support effective behavioral self-regulation. As
part of ongoing research by members of this team, we have developed a
customized smartphone-based decision support tool that facilitates
collection of objectively measured physical activity and dietary
intake data and informs decisions to improve physical activity and
diet choice. This tool is a smartphone platform that provides
real-time, objective feedback on progress toward physical activity and
diet goals, and informed, interactive support from
clinicians. Hardware consists of a smartphone, and a three-axis
accelerometer with Bluetooth capabilities. The interface design
includes touch screen tabs for individuals to review daily diet intake
and physical activity. A patient enters dietary intake into their
smartphone, which is pre-programmed with the Calorie
King\textregistered{} food database to provide a breakdown of
macro-nutrient and micro-nutrient consumption. Physical activity is
captured via the accelerometer. Thus, both physical activity and diet
are captured in near-real time. The smartphone will automatically and
wirelessly send and receive data with no further action required on
the part of the participant. This customized software and technical
platform is currently being used in two randomized controlled trials
in an obese population with support from a grant awarded by the
National Institutes of Health to one of us~\citep{Spring2009} We will
capture data from glucometers upon patient visits to the
Maternal-Fetal Medicine clinic. We will also include assessments of
mood (depression, anxiety, anger), use of cigarettes \& ethanol, and
other relevant health behaviors such as sleep schedule into the
POP-GDM smartphone portal \& web-based portal.

\paragraph{Simulation Trials of POP-GDM vs. Usual care} Using an
approach similar to our prior published work referenced above, we will
compare the performance of POP-GDM and usual care. We will assess
timeliness as the time duration between reception by the clinician of
a message from a device, a patient, or another clinicians and complete
transmission of a response by the clinician to the patient. We will
assess appropriateness according to concordance between the simulated
action initiated by the clinician in response to the message and what
is called for by the consensus recommendations of our expert panel. We
will assess reliability according to the degree of variation in
response time and consistency of simulated action initiated by the
clinician in response to the message. One of us is the director of the
Northwestern Simulation and Immersive Learning facility and will
provide assistance and oversight with these simulation trials.

\paragraph{Expected Outcomes} We expect to build an integrated
platform for management of GDM. We envision POP-GDM as a ``clinician
amplifier'' that will assure orders of nurse practitioners \&
physicians are effectively performed by patients \& staff, that
physical activity, diet, \& health behaviors are effectively
monitored, \& that missing tasks are detected \& remediated. We
anticipate POP-GDM content will be unsurprising to clinicians and
similar to the protocol used by us in an RCT of treatment of mild
GDM. POP-GDM will not replace a clinician but will enhance the
performance of a clinician. We have found that many clinicians keep a
task list on paper or their smartphone or other personal tech
device. Existing task list management tools present little advantage
over paper, as they are decontextualized from the clinical
datastream. One key aspect of POP-GDM is to provide task management
that is linked to the clinical datastream, that adjusts task priority
according to urgency \& importance \& preferences of the task executor
and colleagues, and that integrates in a natural, seamless manner with
patients \& clinicians.

\paragraph{Potential Problems \& Alternative Strategies} We expect to
easily complete development within the first 3 months of our timeline
of the initial version of POP-GDM, suitable for use with patients in
the Maternal-Fetal Medicine clinical research setting. If
unanticipated problems occur, we will scale back the initial POP-GDM
so as to remove those prescription elements not yet ready for
deployment to real patients. We expect to build information transfer
protocols into POP-GDM that will assure interoperable data transfer
from devices \& from the EPICare EHR system. We expect to continue use
of the simulation facilities for development of successive versions of
POP-GDM, as we learn from the experience of using POP-GDM to manage
real patients.

