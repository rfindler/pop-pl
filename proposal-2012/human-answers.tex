\documentclass{article}
\usepackage{times}
\begin{document}

\section{Protection of Human Subjects}
\subsection{Risks to Human Subjects}
\subsubsection{Human Subjects Involvement, Characteristics, and Design}
\begin{itemize}
\item	As outlined in the Research Strategy section, the involvement of human subjects will be involved in two ways. Clinicians (nurses, pharmacists, and physicians) will participate in simulations of scenarios that are enriched for “opportunities for error.” These consist of 90 minutes of managing 6 to 20 simulated patients in Northwestern’s Simulation and Immersive Learning Facility.  In the actual clinical practice portion of the study, the best-practice prescriptions (order sets) will be applied via standard CPOE and as implemented on the new POP–SE (intervention group). 
\item We anticipate that approximately 60 clinicians will participate in these studies. The ages of eligible subjects is 24 to 80 years. This population is drawn from the clinicians that practice in the hospital. 
\item We anticipate that 120 pregnant woman will be recruits to participate as subjects. These patients will be recruited from the maternal-fetal medicine clinic. These patients will participate in the statistical process control trial described in aim 3 of our proposal.
\item We anticipate abstracting data from the electronic health records of approximately 1,500 women with GDM or at risk for GDM.
\item We will recruit volunteers from our own staff for the simulation part of the study. Patients will be recruited for the cliical trials and informed consent obtained prior to their involvement in the study.
\item For the review of electronic medical records, all identifiers will be removed during abstraction of the data.
\item There will be no subjects that are from special populations.
\item We have never detected any harm to subjects in prior simulation studies. The statistical process control study involve some risk of harm as a programming error might lead to inaccurate application of the best practice prescription. However, the point of the study is to reduce error and we expect there to be a strong effect. The overall risk of error is therefore unlikely to be increased.
\item Not applicable
\end{itemize}
\subsubsection{Sources of Materials}
\begin{itemize}
\item No tissue samples or other materials will be used.
\item Describe any data that will be collected from human subjects for the project(s) described in the application. We will be collecting clinical data
\item Only qualified investigators and trained research staff will have access to individually identifiable private information about human subjects.
\item All data containing PIH will be maintained on password-protected servers. Our NU Enterprise Data Warehouse has extensive safeguards to protect PHI. Data will be deidentified in the process of abstraction.
\end{itemize}
\subsubsection{Potential Risks}
\begin{itemize}
\item The risk to the clinicians participating in the simulations is trivial. These activities are essentially the same as the immersive learning activities used for competency training. In the clinical practice part of the study, it is possible that patients might be injured as a consequence of a software error, however, the overall risk of injury is almost certainly lower with the interventions than with the routine process. It is possible that loss of confidentially might occur, leading to psychological, financial, or legal risk. We are taking precautions to prevent this. 
\item Patients declining participation will receive ordinary care. 
\end{itemize} 
\subsection{Adequacy of Protection Against Risks}
\subsubsection{Recruitment and Informed Consent}
\begin{itemize}
\item All the investigators and staff are trained in the ethics of human subject research, require certification of competency to conduct human subject research. Informed consent will be obtained from patients at a time that they are competent to make a decision. We will show a brief video describing the research to each subject. Following this, a staff member will inquire as to willingness to participate in the research and informed consent will be obtained. Neonates will be involved in the gestational diabetes portion of the clinical study, as we measure birthweight and assess obstetric complications as outcome variables. No intervention will be directed at the neonates. All interventions will be on adults, including pregnant women with gestational diabetes mellitus. The goal of intervention here is to assure that best practices for managing diabetes mellitus are conducted. Thus, it is likely that the intervention group will not have a higher risk than the control group.
\item Informed consent will be obtained in the course of ordinary care by a clinical research nurse, physician-investigator, or pharmacist-investigator at a time when the patient is under no duress. Prospective subjects will be informed of the nature of the study, the potential risks, and benefits of participation in the study. Written informed consent will be obtained.
\end{itemize}
\subsubsection{Protections Against Risk}
\begin{itemize}
\item The point of the study is to study an intervention whose purpose is to reduce risk. Thus, extensive testing of the software in simulation will be done prior to exposing patients to the prescription engine. Study personnel will monitor the conduct of the study for any untoward events. This is part of the debugging process. It should, perhaps, be mentioned that routine care involves the same theapeutic intervention without debugging of the prescriptions used. We expect that the intervention will lower the risk of patient injury. We will devise and maintain an operational data integrity plan based on the policies of our hospital. We have devised similar plans with multiple prior studies.
\item Research involving vulnerable populations, as described in the DHHS regulations, Subparts B-D must include additional protections. Refer to DHHS regulations, and OHRP guidance: 
\item Pregnant women, neonates, and fetuses will participate in this study. The intervention has potential to significantly reduce risk of harm to mother and fetus and meets criteria established by each subjpart of 45 CFR. 46.204 regarding research involving pregnant women or fetuses 
\item The interventions in the clinical care portion of the study occur in routine clinical settings, with availability of resources for mitigating patient injury. As this is a clinical trial of a safety intervention, we will closely monitor outcomes to avoid, detect, and appropriately respond to errors, should they occur. When errors occur in the routine care group or the intervention group, then an appropriate response to protect the human subject will be initiated immediately. Our statistician will monitor outcomes and the senior members of each substudy will periodically review adverse events so as to assess whether these are causally related to the intervention. All control and intervention subjects will have full access to the rescue and resuscitation resources routinely provided to patients at Northwestern.
\end{itemize}
\subsection{Potential Benefits of the Proposed Research to Human Subjects and Others}
\begin{itemize}
\item The clinicians may benefit through learning about some simple ways to avoid errors. Those patients in the intervention group may benefit due to lower risk of patient injury.
\item In routine care, there is one error per patient per day. We expect to lower this by an order of magnitude or more.
\end{itemize} 
\subsection{Importance of the Knowledge to be Gained}
\begin{itemize}
\item In our previous work, we designed an analgesia prescription that was widely adopted at a medical center and that eliminated severe or fatal opioid adverse drug events. 
\item The knowledge gained in this study is likely to improve the safety and efficacy of the theapeutic interventions we provide our patients.
\item We will comply with all existing regulations regarding software used in clinical settings, as this regulation evolves.
\end{itemize} 
\subsection{Data and Safety Monitoring Plan}
\begin{itemize}
\item We will abstract and analyze the actual practice data using a statistical process control design that would detect an increased risk of injury should suh an increased risk occur. We have previously monitored interventions of this type in this way.An oversight committee that includes our statistician, clinical pharmacologist, and senior clincians will review the safety data monthly or more often. We report adverse drug events to the IRB on an ongoiring basis. 
We expect risks to be higher in the control group.  
\item f.	 
\end{itemize}
\subsection{ClinicalTrials.gov Requirements}
Public Law 110-85 (also known as the FDA Amendments Act (FDAAA) of 2007) We will register each study with ClinicalTrials.gov. The principal investigator (Dr. Belknap) is responsible for registering the trial.

\end{document}
