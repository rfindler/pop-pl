\subsection{Find a Predictive Statistical Model for Maternal Glycemia \& Obstetric Outcomes}\label{sec:stat}
We will datamine our SPCT \& the Northwestern Electronic Data
Warehouse. Exposure variables will include demographics, mood, health
behaviors, body-mass index, co-morbidities, activity, diet, \&
gestational age. Effect variables will include maternal glycemia,
insulin use, clinician time-burden, birthweight, \& delivery mode.

The work done for the rest of this this proposal will generate
valuable data that can be repurposed to provide insight into the
exposures and effects that drive GDM. The objective of this aim is to
datamine the dataset generated by our RCT of POP-GDM and to supplement
this dataset with data obtained from our institutional clinical data
warehouse. Based on prior similar analyses that we have conducted, we
expect to use hierarchical optimal discriminant analysis (HODA) to
identify a simple, compact, highly-predictive statistical model of
risk of GDM at baseline and over the course of gestation. We have
previously developed several such predictive models, for example
\citet{KyriacouYarnoldBelknap2007}'s work demonstrates one.

\paragraph{Justification \& Feasibility} A particular strength of this
proposal is that we will be using the recently established
Northwestern University Enterprise Data Warehouse (NUEDW), a unique
and valuable resource providing access to data from 1993 through 2010
for both hospitalized and ambulatory patients. In preparation for this
grant, we have demonstrated that we can extract from NUEDW all
required data for this project, including data originally entered in
Cerner's Powerchart EHR for hospitalized patients and data entered
into EPIC's EpicCare EHR for ambulatory patients. One of us is the
statistician who co-invented the Hierarchical Optimal Discriminant
Analysis Statistical Methods that we will use here~\citep{Yarnold2004}.

\paragraph{Research Design} We have considerable experience with the design
and conduct of (HODA) projects. We will repurpose the data generated
by the work in this proposal and obtain additional data by datamining the extensive
NUEDW dataset to identify univariable predictors of GDM and also to
identify a multivariable predictor model of GDM occurrence. One novel
aspect of this analysis will be that we are collecting near-real time
assessments of mood, which we expect may bear on adherence to
prescribed interventions. One of us has previously studied the effects
of mood on type II diabetes mellitus~\citep{Hall2008}. Mood
assessment tools used in this previous work will be used to generate
synthetic variables for input into our multivariable predictor model.

\paragraph{Expected Outcomes} This work will provide an opportunity
for hypothesis testing and hypothesis generation for future
studies. We expect to identify a HODA statistical model that could
serve as a clinical tool that will predict, with a useful degree of
accuracy, which women are at risk to eventually develop GDM. We expect
that this GDM clinical tool will then be validated in future,
separately-funded cohort studies or longitudinal studies or in RCTs to
assess utility of interventions identified by our model as pivotal for
determining risk of GDM. This GDM clinical tool would also be
incorporated into future versions of POP-GDM so as to identify, in a
routine clinical setting, opportunities for reducing risk of GDM in
individual patients. The application of the statistical analysis
methods use the commercial statistics package that we have previously
used and was developed by one of us~\citep{Yarnold2004}.

