\subsection{Establish whether POP-GDM improves
adherence to prescribed interventions, improving maternal glycemia, \&
reducing fetal macrosomia \& need for insulin.} In newly pregnant women
at high risk for GDM, we will conduct a statistical process control
trial (SPCT) to optimize the design \& performance of POP-GDM \&
conduct a randomized clinical trial (RCT) of performance of
POP-GDM. We hypothesize that POP-GDM compared to usual care will
improve adherence to prescribed diet \& activity, improve maternal
glycemia, \& reduce clinician time-burden, fetal macrosomia
(birthweight > 4 kg), \& need for insulin.

To establish that POP-GDM outperforms usual care, we must first
optimize POP-GDM in the clinical setting \& then compare the optimized
POP-GDM to usual care. The objectives of this aim are to optimize
POP-GDM by conducting a statistical process control trial (SPCT) so as to
establish whether POP-GDM outperforms usual care. Our working
hypothesis is that POP-GDM will improve adherence of patients to
prescribed interventions including diet, physical activity, behavior
\& (when necessary) insulin use. We will test our working hypothesis
by using POP-GDM as a “clinician amplifier” that will identify \&
remediate patient non-adherence, provide an efficient communication
channel between patient \& clinician, provide frequent
patient-specific reinforcement of the plan to which patient \&
clinicians have agreed, and assure that critical tasks are performed
as required. From the clinician's perspective, POP-GDM will help to
prioritize tasks according to urgency and importance, \& assure that
missed patient or clinician tasks are appropriately remediated.

\paragraph{Justification \& Feasibility}
 As a consequence of a rise in
maternal obesity \& lower diagnostic threshold, we project that
prevalence of GDM will double to 600,000 women over the three-year
period of this proposed work. A crisis looms, as there are inadequate
resources to manage this projected load of GDM patients. GDM increases
risk of stillbirth, C-section, \& pre-eclampsia,~\citep{Langer2006} \& has
long-term effects on offspring of women with GDM, including increased
risk of impaired glucose tolerance in adult offspring, as reported in
a study led by one of us~\citep{Metzger2007}. Also led by one of us, the
Hyperglycemia and Adverse Pregnancy Outcome (HAPO) study of 25,000
pregnant women showed that adverse outcomes in mother, fetus, \&
neonate increase with rising maternal glycemia at 24-28 weeks, even at
levels below the GDM threshold~\citep{Metzger 2008}. GDM treatment improves
outcomes of both mother \& offspring, with lower perinatal morbidity
and better health-related maternal quality of
life~\citep{Crowther2005}. 
Maternal size, parity, and disease severity is at least 2-fold
higher for untreated women~\citep{Langer2005a}. Treatment of mild GDM
reduces fetal growth, shoulder dystocia, cesarean section, and
hypertensive disorders~\citep{LandonPeaceman2009}. The most appropriate
and effective treatment for GDM is uncertain. We expect that a
technologically-savvy approach will optimize management of GDM and
address this otherwise difficult to address future clinical need.

\paragraph{Research Design}
We have extensive prior experience with conduct of both
SPCTs~\citep{Belknap2008,GraumlichBelknap2000} and, specifically in
this population RCTs~\citep{LandonPeaceman2008}. We are experienced with
development \& application of taxonomies \& metrics for assessing
healthcare process performance \& patient outcome and also with
failure mode, effect, \& criticality analysis. We have previously
developed \& implemented improved instruments \& methods for assessing
adverse events~\citep{Belknap2010} and therapeutic failure. We will use
similar methods and tools to identify \& fix potential bugs in
POP-GDM.

\paragraph{Statistical Process Control Trial of POP-GDM }
The effect measures used
for this SPCT trial of POP-GDM will be rate of adherence to entering
dietary \& activity data into the web-based or smartphone-based
portal, adherence to dietary targets for total calories, fat calories,
\& carbohydrates; \& physical activity as assessed by downloaded
accelerometer data and daily activity reports. The PI was the first to
publish use of the balanced-scorecard methodology to evaluation of a
healthcare improvement project~\citep{Graumlich,Belknap2001}. We will use
a similar balanced scorecard approach here. For this SPCT of POP-GDM,
we will first collect baseline data from 90 patients to serve as controls and then enroll
90 patients in the active intervention period of the SPCT. These women will use POP-GDM from initial identification of eligibility until
delivery. The purpose of this work will be to
identify bugs and improve the algorithms in POP-GDM. Women
will be eligible for this SPCT of POP-GDM if they are pregnant and at
high risk for GDM, as determined by a clinical algorithm based on body
mass index, maternal age, \& avoidance of vigorous exercise. We have
previously applied software engineering \& debugging to a paper
analgesia prescription, the "Patient-oriented Prescription for
Analgesia" (POPA) \& conducted a statistical process control clinical
trial and nested subcohort analysis in a population of 153,260
hospitalized adults. In the orthopedics subcohort, POPA increased
recording of pain scores (94\% vs. 72\%, P $<$ 0.00001) and use of
adjuvant analgesics (95\% vs. 40\%, P $<$ 0.00001) and resulted in
fewer opioid-associated severe ADEs than routine patient-controlled
analgesia (PCA) (0\% vs. 2.7\%, number needed to treat (NNT) = 35, P
$<$ 0.015) (figure on left below) Hospital-wide, POPA use increased to
62\% of opioid prescriptions (diffusion half-life = 98 days), (upper
figure, right, below), while opioid-associated severe/fatal ADEs fell
from an initial peak of seven per month to zero per month during the
final 6 months (P $<$ 0.0016) of the study, (lower figure, right,
below).

 

Figure 1. (a) Optimal classification tree analysis (b) Quarterly hospitalwide purchases of POPA cartridges. (c) A u-type process control run chart (σ = 3) for hospitalwide opioid-associated severe/fatal ADEs. 

\paragraph{SPCT Intervention and Evaluation} The
intervention will consist of monitoring, \& near-real time feedback
for diet \& physical activity, and (if meeting GDM criteria),
near-real time monitoring of blood glucose \& insulin use. These
interventions will be patient-specific and include capability for
asynchronous queries by patients \& advice from clinicians. Monitoring
\& feedback will be provided for POP-GDM via smartphone \& web portal,
with provisions for redundancy via text-messaging, automated voice, \&
patient-clinician phone discussion. The criteria for GDM and treatment
for GDM are currently undergoing revision based on recent results on
risk and treatment; one of us is closely involved with this update of
GDM guidelines~\citep{Metzger2010}. These revised guidelines will be
available by this project's start date; we will use the updated
guidelines to specify content of POP-GDM. Based on current patterns at
Prentice Womens Hospital \& Maternity Center of Northwestern Memorial
Hospital, we project there to be 375 women with actual GDM over the
2.5 year period of this RCT and we expect that 25\% will be enrolled
in this RCT. The primary outcome variable for this SPCT of POP-GDM will
be adherence to prescribed interventions; measured as follows: for
diet, percentage of days within 80\% to 100\% of dietary caloric
intake, using Calorie King® as the nutritional database and as
determined from data entered through the smartphone portal or web
portal or redundant input for physical activity, percentage of days at
100\% to 200\% of prescribed physical activity as determined from data
downloaded from accelerometer and/or data recorded into the POP-GDM
web portal or smartphone portal. The criteria for non-adherence will
be less than 50\% adherence to the dietary prescription or less than
25\% adherence to the physical activity prescription. Secondary
outcome measures will include maternal glycemia, birthweight, \& need
for insulin or oral diabetes drug. We will also assay plasma
creatinine, hepatic transaminases \& bilirubin, magnesium, C-reactive
protein, insulin, \& adiponectin at baseline and 10, 20, \& 30 weeks
of gestation.

One of us was previously an investigator for a multi-center RCT in
women with mild GDM in the 24th to 31st week of
gestation~\cite{LandonPeaceman2009}. In this study, 958 woman with an
abnormal glucose tolerance test but with a fasting glucose level below
95 mg/dL were randomly assigned to a control group (usual prenatal
care) or to treatment group (dietary intervention, self-monitoring of
blood glucose, and (if necessary) insulin therapy. The study found
significant reductions with treatment as compared with usual care in
mean birth weight (3302 vs. 3408 g), neonatal fat mass (427 vs. 464
g), the frequency of large-for-gestational- age infants (7.1\%
vs. 14.5\%), birth weight greater than 4000 g (5.9\% vs. 14.3\%),
shoulder dystocia (1.5\% vs. 4.0\%), and cesarean delivery (26.9\%
vs. 33.8\%). Treatment of GDM, as compared with usual care, was also
associated with reduced rates of preeclampsia and gestational
hypertension (combined rates for the two conditions, 8.6\% vs. 13.6\%;
P=0.01). We and our support staff are very experienced with such
clinical trials in this population.  2.3.4 Expected Outcomes We expect
to validate that POP-GDM improves adherence to prescribed dietary \&
physical activity interventions in women at high risk for GDM. It is
possible that we might also demonstrate that POP-GDM is superior to
usual care for maternal glycemia \& birthweight, and that it is
superior to usual care for avoiding a need for insulin. Upon
successful completion of this work, we expect to conduct a
multi-center RCT to evaluate POP-GDM as a means of reducing obstetric
complications, such as need for Caesarian Section.
 

\paragraph{Potential Problems \& Alternative Strategies}
We expect to easily meet our enrollment goal for this study, as we
project 375 women with GDM will be seen in the Maternal-Fetal Medicine
Clinic at Northwestern Prentice over the 2.5 year enrollment period
for this study. Approximately three patients at high-risk for GDM
would be expected for each actual patient with GDM, giving a cohort
pool size of 1,125 potential subjects over this interval, we would
meet our enroll target of 90 women for the intervention phase of the SPCT with as low as a
12\% enrollment rate, allowing for dropout and unevaluability. If we
were to have unanticipated problems with enrollment or retention, then
we would still expect to enroll and retain the minimum 24 subjects
needed to test our primary outcome variable.

