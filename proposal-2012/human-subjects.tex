\section{Protection of Human Subjects}

Human subjects will be involved in three ways. Approximately 60 clinicians (nurses, pharmacists, and physicians) will participate
in simulations of scenarios that are enriched for “opportunities for error.” These consist of 90 minutes of managing 6 to 20 simulated patients
in Northwestern’s Simulation and Immersive Learning Facility. In the actual clinical practice portion of the study, the best-practice prescriptions (order sets) 
will be applied via standard CPOE and as implemented on the new POP–SE (intervention group). We will also enroll  human subjects who are pregnant women, recruited
from the maternal-fetal medicine clinic and from obstetrics practices
in the hospital. We anticipate that 90 pregnant woman will be
recruited to participate as subjects. These subjects will participate
in the statistical process control trial from
section~\ref{sec:popgdm}.

We also anticipate abstracting data from the electronic health records
of approximately 1,800 patients at Northwestern with GDM or at risk
for GDM.  For the review of electronic medical records, all
identifiers will be removed during abstraction of the data.  We will
not be collecting tissue samples, only clinical data.

\subsection{Potential Risks and Protection Against Them}

We have never detected any harm to
subjects in prior simulation studies. In the clinical research studies, our main research aim is to lower risk by improving
the best practices of care for patients with GDM. Thus, we expect that the overall risk of error is unlikely to increase, as compared to
standard care.  That said, it is possible
that patients might be injured as a consequence of an error in
POP-GDM.  We will mitigate this risk through extensive testing and
simulation of POP-GDM, as well as extensive testing of the underlying
\poppl{} language.  During the actual clinical trial phase of the
investigation, we will closely monitor outcomes to avoid, detect, and
appropriately respond to errors, should they occur. Our statistician will monitor outcomes and the senior
members of each sub-study will periodically review adverse events so as
to assess whether these are causally related to the intervention. All
control and intervention subjects will have full access to the rescue
and resuscitation resources routinely provided to patients at
Northwestern.

Neonates will be involved in the study, as we measure birthweight and
assess obstetric complications as outcome variables, but intervention
will be directed at them.
We will follow the \citet{45cfr46}'s guidelines 45 CFR 46.204
on research involving pregnant women or fetuses.

It is also possible that loss of confidentially might occur, leading
to psychological, financial, or legal risk.  To mitigate this risk,
only qualified investigators and trained research staff will have
access to individually identifiable private information about human
subjects.  Also, all data containing private health information will
be maintained on password-protected servers. Our NU Electronic Data
Warehouse has extensive safeguards to protect patient
information. Data will be de-identified in the process of abstraction.

Informed consent will be obtained from patients at a time that they
are competent to make a decision and at a time when the patient is
under no duress. We will show a brief video describing the research to
each subject. Prospective subjects will be informed of the nature of
the study, the potential risks, and benefits of participation in the
study.  Following this, a clinical staff member will inquire as to willingness to participate
in the research and, if consent is given, it will be recorded in
written form. Patients declining participation will receive ordinary
care.
Finally, in compliance with Public Law 110-85 (also known as the FDA
Amendments Act of 2007), we will register each study with
ClinicalTrials.gov. Dr.\ Belknap takes responsibility for registering
the trial.

\subsection{Potential Benefits and Importance of the Knowledge to be Gained}

In the short-term, we expect that POP-GDM will lower the risks
associated with GDM at Northwestern and eventually POP-GDM (or its
successors) will be adopted by hospitals around the world, lowering
risks everywhere. Longer term, we expect the synergy of computers and
trained clinical professionals to result in a dramatic reduction of
errors in hospitals. In routine care at top-notch hospitals there is,
on average, one error per patient per day. We expect to lower this by
an order of magnitude or more.
